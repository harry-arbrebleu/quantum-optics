\documentclass{report}
\usepackage{luatexja}
\usepackage{amsmath, amssymb, type1cm, amsfonts, latexsym, mathtools, bm, amsthm, url, color}
\usepackage{multirow, hyperref, longtable, dcolumn, tablefootnote, physics}
\usepackage{tabularx, footmisc, colortbl, here, usebib, microtype}
\usepackage{graphicx, luacode, caption, fancyhdr}
\usepackage[top = 20truemm, bottom = 20truemm, left = 20truemm, right = 20truemm]{geometry}
\usepackage{ascmac, fancybox, color, tabularray, subcaption}
\usepackage{luatexja-fontspec, multicol}
\usepackage{upgreek, colortbl, mhchem}
\usepackage{biblatex, array, truthtable}
\usepackage{listings, jvlisting}
\usepackage{xcolor, siunitx, float, dcolumn}
\sisetup{
  table-format = 1.5, % 小数点以下の桁数を指定
  table-number-alignment = center, % 数値の中央揃え
}
% \abovedisplayskip = 0pt
% \belowdisplayskip = 0pt
\allowdisplaybreaks
% \DeclarePairedDelimiter{\abs}{\lvert}{\rvert}
\newcolumntype{t}{!{\vrule width 0.1pt}}
\newcolumntype{b}{!{\vrule width 1.5pt}}
\UseTblrLibrary{amsmath, booktabs, counter, diagbox, functional, hook, html, nameref, siunitx, varwidth, zref}
\setlength{\columnseprule}{0.4pt}
\captionsetup[figure]{font = bf}
\captionsetup[table]{font = bf}
\captionsetup[lstlisting]{font = bf}
\captionsetup[subfigure]{font = bf, labelformat = simple}
\setcounter{secnumdepth}{5}
\newcolumntype{d}{D{.}{.}{5}}
\newcolumntype{M}[1]{>{\centering\arraybackslash}m{#1}}
\everymath{\displaystyle}
\DeclareMathOperator*{\AND}{\cdot}
\DeclareMathOperator*{\NAND}{NAND}
\DeclareMathOperator*{\NOT}{NOT}
\DeclareMathOperator*{\OR}{+}
% \DeclareMathOperator{\grad}{grad}
% \DeclareMathOperator{\div}{div}
% \DeclareMathOperator{\rot}{rot}
\renewcommand{\i}{\mathrm{i}}
% \newcommand{\NOT}[1]{\overline{#1}}
\newcommand{\rot}{\curl}
\renewcommand{\hat}[1]{\overhat{#1}}
\renewcommand{\thesubfigure}{(\alph{subfigure})}
\newcommand{\m}[3]{\multicolumn{#1}{#2}{#3}}
\renewcommand{\r}[1]{\mathrm{#1}}
\newcommand{\e}{\mathrm{e}}
\newcommand{\Ef}{E_{\mathrm{F}}}
\renewcommand{\c}{\si{\degreeCelsius}}
\renewcommand{\d}{\r{d}}
\renewcommand{\t}[1]{\texttt{#1}}
\newcommand{\kb}{k_{\mathrm{B}}}
\renewcommand{\phi}{\varphi}
\renewcommand{\laplacian}{\grad^2}
\newcommand{\reff}[1]{\textbf{図\ref{#1}}}
\newcommand{\reft}[1]{\textbf{表\ref{#1}}}
\newcommand{\refe}[1]{\textbf{式\eqref{#1}}}
\newcommand{\refp}[1]{\textbf{コード\ref{#1}}}
\newcommand{\refa}[1]{\textbf{\ref{#1}}}
\renewcommand{\lstlistingname}{コード}
\renewcommand{\theequation}{\thesection.\arabic{equation}}
\renewcommand{\footrulewidth}{0.4pt}
\newcommand{\mar}[1]{\textcircled{\scriptsize #1}}
\newcommand{\combination}[2]{{}_{#1} \mathrm{C}_{#2}}
\newcommand{\thline}{\noalign{\hrule height 0.1pt}}
\newcommand{\bhline}{\noalign{\hrule height 1.5pt}}
\newcommand*{\myCurrentTime}{
  \directlua{ my_current_time() }
}
\newcommand{\Rnum}[1]{
  \ifnum #1 = 1
    I
  \fi
  \ifnum #1 = 2
    I\hspace{-1.2pt}I
  \fi
  \ifnum #1 = 3
    I\hspace{-1.2pt}I\hspace{-1.2pt}I
  \fi
  \ifnum #1 = 4
    I\hspace{-1.2pt}V
  \fi
  \ifnum #1 = 5
    V
  \fi
  \ifnum #1 = 6
    V\hspace{-1.2pt}I
  \fi
  \ifnum #1 = 7
    V\hspace{-1.2pt}I\hspace{-1.2pt}I
  \fi
  \ifnum #1 = 8
    V\hspace{-1.2pt}I\hspace{-1.2pt}I\hspace{-1.2pt}I
  \fi
  \ifnum #1 = 9
    I\hspace{-1.2pt}X
  \fi
  \ifnum #1 = 10
    X
  \fi
}
\newcommand{\cover}{
  \renewcommand{\arraystretch}{3}
  \title{物理情報工学実験報告書}
  \date{}
  \author{}
  \maketitle
  \begin{table}[H]
    \begin{flushright}
      2024年度
    \end{flushright}
    \begin{center}
      \begin{tabularx}{150mm}{|>{\centering}p{40mm}|>{\centering}p{25mm}|>{\centering}p{30mm}|>{\centering\arraybackslash}X|}
        \hline
        \Large{実験テーマ} & \multicolumn{3}{c|}{\Large{A1(直流安定化電源)}} \\ \hline
        \Large{担当教員名} & \multicolumn{3}{c|}{\Large{塚田孝祐}} \\ \hline
        \Large{実験整理番号} & \Large{002} & \Large{実験者氏名} & \Large{青木\ 陽}\\ \hline
        \Large{共同実験者氏名} & \multicolumn{3}{c|}{} \\ \hline
        \Large{曜日組} & \Large{火1班} & \Large{実験日} & \Large{6月25日} \\ \hline
        \Large{実験回} & \Large{9} & \Large{報告書提出日} & \Large{\myCurrentTime}\\ \hline
      \end{tabularx}
    \end{center}
  \end{table}
  \thispagestyle{empty} 
  \addtocounter{page}{-1}
  \clearpage
  \renewcommand{\arraystretch}{1.0}
}
\pagestyle{fancy}
\chead{量子光学ノート}
\rhead{}
\cfoot{\thepage}
\lhead{}
\rfoot{\t{harry\_arbrebleu}}
\setcounter{tocdepth}{4}
\makeatletter
\@addtoreset{equation}{subsection}
\makeatother
\begin{luacode*}
  function my_current_time()
    local date = os.date("*t")
    local year = date.year
    local month = date.month
    local day = date.day
    local hour = date.hour
    local min = date.min
    local sec = date.sec
    local formatted_date = string.format("%d月%d日", month, day)
    tex.sprint(formatted_date)
  end
\end{luacode*}
\lstset{
  language = Matlab, % Set the language for the code
  basicstyle = {\ttfamily},
  identifierstyle = {\small},
  commentstyle = \color{red},
  keywordstyle = \color{blue},
  ndkeywordstyle = {\small},
  stringstyle = \color{orange},
  frame={tb},
  breaklines = true,
  columns=[l]{fullflexible},
  xrightmargin = 5mm,
  xleftmargin = 5mm,
  numberstyle = {\ttfamily\scriptsize},
  stepnumber = 1,
  numbersep = 1mm,
  lineskip = -0.5ex,
  showstringspaces = false,
  numbers = left,
  frame = lines,
  backgroundcolor = \color{gray!10},
  rulecolor = \color{black!30},
}

\definecolor{mygray}{rgb}{0.5,0.5,0.5}
\definecolor{mymauve}{rgb}{0.58,0,0.82}
\definecolor{mygreen}{rgb}{0,0.6,0}

\lstset{ %
  backgroundcolor=\color{white},   % 背景色
  basicstyle=\ttfamily\footnotesize, % 基本の書体スタイル
  breakatwhitespace=false,        % 空白で行分割しない
  breaklines=true,                % 長い行は分割する
  captionpos=b,                   % キャプションの位置
  commentstyle=\color{mygreen},   % コメントのスタイル
  extendedchars=true,             % 非 ASCII 文字をサポート
  frame=single,                   % フレームの表示
  keywordstyle=\color{blue},      % キーワードのスタイル
  language=[LaTeX]TeX,            % 言語を LaTeX に設定
  numbers=left,                   % 行番号を左側に表示
  numbersep=5pt,                  % 行番号とコードの間の距離
  numberstyle=\tiny\color{mygray}, % 行番号のスタイル
  rulecolor=\color{black},        % 枠線の色
  showspaces=false,               % スペースを表示しない
  showstringspaces=false,         % 文字列内のスペースを表示しない
  showtabs=false,                 % タブを表示しない
  stepnumber=1,                   % 行番号を表示する間隔
  stringstyle=\color{mymauve},    % 文字列のスタイル
  tabsize=2,                      % タブの幅
  title=\lstname                  % タイトル
}
\lstset{
  language = C++, % Set the language for the code
  basicstyle = {\ttfamily},
  identifierstyle = {\small},
  commentstyle = \color{red},
  keywordstyle = \color{blue},
  ndkeywordstyle = {\small},
  stringstyle = \color{orange},
  frame={tb},
  breaklines = true,
  columns=[l]{fullflexible},
  xrightmargin = 5mm,
  xleftmargin = 5mm,
  numberstyle = {\ttfamily\scriptsize},
  stepnumber = 1,
  numbersep = 1mm,
  lineskip = -0.5ex,
  showstringspaces = false,
  numbers = left,
  frame = lines,
  backgroundcolor = \color{gray!10},
  rulecolor = \color{black!30},
}
\title{量子光学ノート}
\date{\today}
\author{\t{harry\_arbrebleu}}
\addbibresource{ref.bib}
\defbibheading{bunken}[\refname]{\section*{#1}}
\begin{document}
  \maketitle
  \tableofcontents
  \chapter{Gaussビームの振る舞い}
    ここでは,Maxwell方程式に従う電磁波のうち,強度分布がGaussビームになるものについて考える.
    Gaussビームは強度分布がGauss分布に従う電磁波である.
    以下の議論により,そのような電磁波は直進することが分かる.
    \section{物質中のMaxwell方程式}
      物質中のMaxwell方程式は,
      \begin{align}
        \div \bm{E} &= \frac{\rho}{\epsilon_r\epsilon_0} \label{maxwell-1} \\ 
        \div \bm{B} &= 0 \label{maxwell-2} \\
        \rot \bm{E} &= -\pdv{\bm{B}}{t}\label{maxwell-3} \\
        \rot \bm{B} &= \mu_r\mu_0\bm{i} + \mu_r\epsilon_r\mu_0\epsilon_0\pdv{\bm{E}}{t}\label{maxwell-4}
      \end{align}
      ただし,$\epsilon_0$,$\mu_0$は真空中の誘電率と真空中の透磁率,$\epsilon_r$,$\mu_r$は物質の比誘電率と物質の比透磁率である.
      また,$\bm{E}$,$\bm{B}$は電場と磁場,$\bm{i}$は電流密度,$\rho$は電荷密度である.
      $\rho = 0$,$\bm{i} = 0$のときを考えると,
      \begin{align}
        \div \bm{E} &= 0 \label{maxwell-5} \\ 
        \div \bm{B} &= 0 \label{maxwell-6} \\
        \rot \bm{E} &= -\pdv{\bm{B}}{t} \label{maxwell-7} \\
        \rot \bm{B} &= \mu_r\epsilon_r\mu_0\epsilon_0\pdv{\bm{E}}{t}\label{maxwell-8}
      \end{align}
      となる.
      \refe{maxwell-7}に左から$\rot$を作用させると,
      \begin{align}
        &\rot \qty(\rot \bm{E}) = -\pdv{t}\qty(\rot \bm{B}) \\
        &\Leftrightarrow -\laplacian \bm{E} = -\pdv{t}\qty(\rot \bm{B}) \label{maxwell-9}
      \end{align}
      となる.\refe{maxwell-9}に\refe{maxwell-8}を代入すると,
      \begin{align}
        \laplacian \bm{E} - \frac{n^2}{c^2}\pdv[2]{\bm{E}}{t} = 0\label{maxwell-10}
      \end{align}
      となる.ただし,$n = \sqrt{\epsilon_r\mu_r}$,$c = 1 / \sqrt{\epsilon_0\mu_0}$である.
    \section{Gaussビームでの電場の表式}
      仮定より,今考えている電磁波はGaussビームであるから円筒座標系$(r, \theta, z)$を用いて,\refe{maxwell-10}を議論する.
      $\bm{E}$が$z$軸正方向に進行する波とすると,
      \begin{align}
        E(r, \theta, z, t) = E_0(r, z)\exp\qty(i(kz - \omega t)) \label{maxwell-11}
      \end{align}
      と書ける.ただし,$k = 2\pi / \lambda = n\omega/c$である.%\refe{maxwell-11}が$\theta$に対して変化しないこと
      \subsection{ラプラシアンの計算}
        \refe{maxwell-11}のラプラシアンについて.
        \begin{align}
          \laplacian &= \pdv[2]{x} + \pdv[2]{y} + \pdv[2]{z} 
        \end{align}
        である.$(x, y, z)$から$(r, \theta, z)$への変換を考える.チェーンルールにより,
        \begin{align}
          \pdv{x} = \pdv{r}{x}\pdv{r} + \pdv{\theta}{x}\pdv{\theta} \\ 
          \pdv{y} = \pdv{r}{y}\pdv{r} + \pdv{\theta}{y}\pdv{\theta} \\
        \end{align}
        である.$\pdv[2]{E}{x}$について考える.
        \begin{align}
          \pdv[2]{E}{x} &= \pdv{x}\qty(\pdv{E}{x}) \\
          &= \qty{\pdv{r}{x}\pdv{r} + \pdv{\theta}{x}\pdv{\theta}}\qty{\pdv{r}{x}\pdv{E}{r} + \pdv{\theta}{x}\pdv{E}{\theta}} \\
          &\text{\refe{maxwell-11}が$\theta$に対して変化しないことに注意すると,} \\ 
          &= \pdv{r}{x}\pdv{r}\qty(\pdv{r}{x}\pdv{E}{r}) + \pdv{\theta}{x}\pdv{\theta}\qty(\pdv{r}{x}\pdv{E}{r})
        \end{align}
        となる.同様の計算により,
        \begin{align}
          \laplacian E = &\pdv[2]{E}{x} + \pdv[2]{E}{y} + \pdv[2]{E}{z} \\ 
          = &\pdv{r}{x}\pdv{r}\qty(\pdv{r}{x}\pdv{E}{r}) + \pdv{\theta}{x}\pdv{\theta}\qty(\pdv{r}{x}\pdv{E}{r}) \notag \\ 
          &+ \pdv{r}{y}\pdv{r}\qty(\pdv{r}{y}\pdv{E}{r}) + \pdv{\theta}{y}\pdv{\theta}\qty(\pdv{r}{y}\pdv{E}{r}) \notag \\ 
          &+ \pdv[2]{E}{z} \label{laplacian}\\
        \end{align}
        である.座標変換を考えると,
        \begin{align}
          r^2 &= x^2 + y^2 \\ 
          \frac{y}{x} &= \tan\theta 
        \end{align}
        であるから,
        \begin{align}
          &\pdv{r}{x} = \cos \theta, \pdv{r}{y} = \sin \theta \label{div-x-y-r-t-1} \\
          &\pdv{\theta}{x} = -\frac{\sin \theta}{r}, \pdv{\theta}{y} = \frac{\cos \theta}{r}\label{div-x-y-r-t-2}
        \end{align}
        である.\refe{div-x-y-r-t-1}と\refe{div-x-y-r-t-2}を\refe{laplacian}に代入すると,
        \begin{align}
          \laplacian E &= \pdv[2]{E}{r} + \frac{1}{r}\pdv{E}{r} + \pdv[2]{E}{z} \label{laplacian-2}
        \end{align}
        を得る.\refe{laplacian-2}を\refe{maxwell-10}に用いて,\refe{laplacian}の表記を用いると,
        \begin{align}
          \pdv[2]{E_0}{r} + \frac{1}{r}\pdv{E_0}{r} + \pdv[2]{E_0}{z} + 2\i\pdv{E_0}{z} = 0
        \end{align}
        となる.$\pdv[2]{E_0}{z}$が他の項に比べて小さく無視できるとしたとき,
        \begin{align}
          \pdv[2]{E_0}{r} + \frac{1}{r}\pdv{E_0}{r} + 2\i\pdv{E_0}{z} = 0 \label{maxwell-electro-field}
        \end{align}
        となり,種々の条件を加えて\refe{maxwell-electro-field}について解く.
      \subsection{一様媒質中でのGaussビーム}
        電磁波が屈折率$n$で一様な媒質中を伝播するとする.
        このとき,$E_0(r, z)$は,
        \begin{align}
          E_0(r, z) = \exp\qty(\i P(z) + \frac{\i k}{2q(z)}r^2) \label{condition-1-e0}
        \end{align}
        と書ける.\refe{condition-1-e0}を\refe{maxwell-electro-field}に代入する.
        \begin{align}
          \pdv[2]{E_0}{r} &= \exp\qty(\i P(z))\pdv{r}\qty[\frac{\i k}{2q(z)}\exp\qty(\frac{\i k}{2q(z)}r^2)] \\
          &= \qty[\qty(\frac{\i k}{2q(z)})^2r^2 + \frac{\i k}{2q(z)}]\exp\qty(\i P(z) + \frac{\i k}{2q(z)}r^2) \\
          \frac{1}{r}\pdv{E_0}{r} &= \frac{\i k}{q(z)}\exp\qty(\i P(z) + \frac{\i k}{2q(z)}r^2) \\ 
          2\i k\pdv{E_0}{z} &= 2\i k\qty[\i \dv{p}{z} + \frac{\i kr^2}{2}\dv{z}\qty(\frac{1}{q(z)})]\exp\qty(\i P(z) + \frac{\i k}{2q(z)}r^2) 
        \end{align}
        となる.\refe{maxwell-electro-field}に代入して,$r^2$の係数を比較すると,
        \begin{align}
          \begin{dcases}
            \qty(\frac{1}{q(z)})^2 + \dv{z}\qty(\frac{1}{q(z)}) = 0 \\
            \dv{p}{z} = \frac{\i}{q(z)}
          \end{dcases}\label{condition-2}
        \end{align}
        を得る.
        \refe{condition-2}を解くと,
        \begin{align}
          \begin{dcases}
            q(z) = z + q_0 \\
            p(z) = \i \ln\qty(1 + \frac{z}{q_0})
          \end{dcases}
        \end{align}
        を得る.
      % \subsection{一様でない媒質中を伝搬するとき}
    \section{}
  \appendix
\end{document}